\section{Equilibria and stability}
Equilibrium solution of $\dot{\vec x} = \vec f(\vec x)$ is a solution $\vec x^\ast \in \mathbb R^n$ which is constant, i.e. $\vec f(\vec x^\ast) = \vec 0$. $\vec x^\ast$ is called a fixed point, stationary point, rest point, critical point or steady state.

For maps, $\vec x_{k + 1} = \vec F(\vec x_k)$, the equilibrium solution is a solution $\vec x^\ast \in \mathbb R^n$ such that $\vec x^\ast = \vec F(\vec x^\ast)$.

\begin{definition}
An equilibrium point $\vec x^\ast$ is said to be stable if, given $\epsilon > 0$, there exists a real number $\delta > 0$ such that for any other solution $\vec y(t)$ satisfying $|\vec y(0) - \vec x^\ast| < \delta$, $|\vec y(t) - \vec x^\ast| < \epsilon$ for all $t \geq 0$. Otherwise it is called unstable.
\end{definition}

\begin{definition}
	An equilibrium point $\vec x^\ast$ is said to be asymptotically stable if it is stable and there is a $b > 0$ such that if $|\vec y(0) - \vec x^\ast| < b$ then $\lim_{t \to \infty} |\vec y(t) - \vec x^\ast| = 0$.
\end{definition}

\begin{definition}
	An equilibrium point $\vec x^\ast$ is said to be exponentially stable if it is asymptotically stable and there exist finite $\alpha, \beta, \delta > 0$ such that if $|\vec y(0) - \vec x^\ast| < \delta$ then $|\vec y(t) - \vec x^\ast| \leq \alpha e^{-\beta t} |\vec y(0) - \vec x^\ast|$ for $t \geq 0$.
\end{definition}

\subsection{Analysing stability in 2x2 systems}
In a linear system
\begin{align*}
	\dot{\vec x} &= \vec A \vec x \\
	\vec x(0)	 &= \vec x_0 
\end{align*}
where $\vec x \in \mathbb R^2$ where we can decompose $\vec A = \vec W \vec \Lambda \vec W^{-1}$ according to Corollary~\ref{cor:la:linear}, we can use the coordinate transformation $\vec z = \vec W^{-1} \vec x$ and $\vec z_0 = \vec W^{-1} \vec x_0$ to analyse the system in the transformed coordinates $\vec z$:
\begin{align*}
	\dot{\vec z} &= \vec \Lambda \vec z \\
	\vec z(0)	 &= \vec z_0
\end{align*}
hence according to Theorem~\ref{thm:lec1:linear}:
\begin{equation*}
	\vec z(t) = e^{\vec \Lambda t} \vec z_0
\end{equation*}

We now analyse the three possible cases of $\vec \Lambda$:

\paragraph{Real, distinct eigenvalues.} We have for $\lambda, \mu, \lambda \neq \mu$:
\begin{equation*}
	\vec \Lambda = 
	\begin{bmatrix}
		\lambda & 0 \\
		0 & \mu
	\end{bmatrix}
\end{equation*}
and
\begin{equation*}
	e^{\vec \Lambda t} =
		\begin{bmatrix}
			e^{\lambda t}	& 0 \\
			0 				& e^{\mu t}
		\end{bmatrix}
\end{equation*}
hence
\begin{align*}
	\vec z(t) 	&= 
		\begin{bmatrix}
			e^{\lambda t}	& 0 \\
			0 				& e^{\mu t}
		\end{bmatrix}
		\vec z_0 \\
				&= 
		\begin{bmatrix}
			z_{1, 0} e^{\lambda t} \\
			z_{2, 0} e^{\mu t}
		\end{bmatrix}
\end{align*}

There can be three subcases:
\begin{itemize}
	\item $\lambda < \mu < 0$: stable.
	\item $\lambda < 0 < \mu$: saddle shape (separatrices).
	\item $0 < \lambda < \mu$: unstable (opposite arrows to stable case).
\end{itemize}

\paragraph{Duplicate real eigenvalue.} We have for $\lambda \neq 0$:
\begin{equation*}
	\vec \Lambda = 
	\begin{bmatrix}
		\lambda & 1 \\
		0 & \lambda
	\end{bmatrix}
\end{equation*}
and
\begin{equation*}
	e^{\vec \Lambda t} = 
		e^{\lambda t}
		\begin{bmatrix}
			1 	& t \\
			0 	& 1
		\end{bmatrix}
\end{equation*}
hence
\begin{align*}
	\vec z(t) 	&= 
		e^{\lambda t}
		\begin{bmatrix}
			1 	& t \\
			0 	& 1
		\end{bmatrix}
		\vec z_0
\end{align*}

There can be two subcases:
\begin{itemize}
	\item $\lambda < 0$: stable.
	\item $\lambda > 0$: unstable.
\end{itemize}

\paragraph{Complex eigenvalues.} We have for $\lambda_1 = a + ib, \lambda_2 = a - ib$:
\begin{equation*}
	\vec \Lambda = 
	\begin{bmatrix}
		a & -b \\
		b & a
	\end{bmatrix}
\end{equation*}
and
\begin{equation*}
	e^{\vec \Lambda t} =
		e^{at}
		\begin{bmatrix}
			\cos{bt}	& -\sin{bt} \\
			\sin{bt}	& \cos{bt}
		\end{bmatrix}
\end{equation*}
hence
\begin{align*}
	\vec z(t) 	&= 
		e^{at}
		\begin{bmatrix}
			\cos{bt}	& -\sin{bt} \\
			\sin{bt}	& \cos{bt}
		\end{bmatrix}
		\vec z_0
\end{align*}

We can look at the norm of $\vec z(t)$ to analyse the shape of the solution:
\begin{align*}
	\| \vec z(t) \|^2	&= |e^{at}|^2 \left\|
							\begin{bmatrix}
								z_{1, 0}\cos{bt} - z_{2, 0}\sin{bt} \\
								z_{1, 0}\sin{bt} + z_{2, 0}\cos{bt}
							\end{bmatrix} \right\|^2 \\
						&= |e^{at}|^2 (z_{1, 0}^2\cos^2{bt} - 2z_{1, 0}z_{2, 0}\sin{bt}\cos{bt} + z_{2, 0}^2\sin^2{bt} + \\
							&\,\,\,z_{1, 0}^2\sin^2{bt} + 2z_{1, 0}z_{2, 0}\sin{bt}\cos{bt} + z_{2, 0}^2\cos^2{bt}) \\
						&= |e^{at}|^2 (z_{1, 0}^2 (\sin^2{bt} + \cos^2{bt}) + z_{2, 0}^2 (\sin^2{bt} + \cos^2{bt})) \\
						&= |e^{at}|^2 (z_{1, 0}^2 + z_{2, 0}^2) \\
						&= |e^{at}|^2 \| \vec z_0 \|^2
\end{align*}
i.e. $\|\vec z(t)\| = |e^{at}| \|\vec z_0\|$. So
\begin{itemize}
	\item $a = 0$: circle.
	\item $a < 0$: stable spiral.
	\item $a > 0$: unstable spiral.
	\item $b > 0$: anticlockwise.
	\item $b < 0$: clockwise.
\end{itemize}

\subsection{Nonlinear systems}
Consider $\dot{\vec x} = \vec f(\vec x)$ with an equilibrium $\vec x^\ast$, such that $\vec f(\vec x^\ast) = \vec 0$. Let $\vec x = \vec x^\ast + \vec w$ and assume $\vec f$ is differentiable and not that the equilibrium point doesn't change with time. Then for $j = 1, \dotsc, n$:
\begin{equation*}
	f_j(\vec x + \vec w) = f_j(\vec x^\ast) + (\grad_{\vec x} f_j(\vec x^\ast))^T \vec w + \cdots
\end{equation*}

Hence we can write
\begin{equation*}
	\dot{\vec x} = \dot{\vec w} = D\{\vec f(\vec x^\ast)\} \vec w + \cdots
\end{equation*}
where
\begin{equation*}
	D\{\vec f(\vec x^\ast)\} =
		\begin{bmatrix}
			(\grad_{\vec x} f_1(\vec x^\ast))^T \\
			\vdots \\
			(\grad_{\vec x} f_n(\vec x^\ast))^T \\
		\end{bmatrix} \in \mathbb R^{n \times n}
\end{equation*}
Hence we can analyse the following small perturbation model
\begin{equation*}
	\dot{\vec w} \approx D\{\vec f(\vec x^\ast)\} \vec w = \vec A \vec w
\end{equation*}

\begin{definition}
	Let $\vec x^\ast$ be an equilibrium of $\dot{\vec x} = \vec f(\vec x)$. Then $\vec x^\ast$ is called a hyperbolic fixed point if none of the eigenvalues of $D\{\vec f(\vec x^\ast)\}$ have zero real part.
\end{definition}

\begin{theorem}
	Suppose $\vec x^\ast$ is a hyperbolic fixed point and all the real parts of the eigenvalues are negative. Then the equilibrium solution $\vec x = \vec x^\ast$ is asymptotically stable.
\end{theorem}