\section{Lyapunov Functions}
\begin{theorem}
	Consider a nonlinear system of the form
	\begin{equation*}
		\dot{\vec x} = \vec f(\vec x)
	\end{equation*}
	and let $\vec x^\ast$ be an equilibrium point, i.e. $\vec f(\vec x^\ast) = \vec 0$.
	Let $V: D \to \mathbb R$ (unrelated to $V$ in conservative systems) be a continuously differentiable function, defined on a neighbourhood $D$ of (open set containing) $\vec x^\ast$ such that
	\begin{itemize}
		\item $V(\vec x^\ast) = 0$ and $V(\vec x) > 0$ for $\vec x \neq \vec x^\ast$.
		\item $\dot V(\vec x) = (\grad_{\vec x} V(\vec x))^T \vec f(\vec x) \leq 0$ for all $\vec x \in D \setminus \{\vec x^\ast\}$.
	\end{itemize}
	Then $\vec x^\ast$ is stable. If moreover
	\begin{itemize}
		\item $\dot V(\vec x) < 0$ in $D \setminus \{\vec x^\ast\}$.
	\end{itemize}
	then $\vec x^\ast$ is asymptotically stable.
\end{theorem}

\subsection{Vector fields possesing an integral}
\subsection{Hamiltonian Systems}
Let $\vec x, \vec y \in \mathbb R^n$, and $H = H(\vec x, \vec y)$. A system of the form
\begin{align*}
	\dot{\vec x} &= \frac{\partial H}{\partial \vec y} \\
	\dot{\vec y} &= -\frac{\partial H}{\partial \vec x},
\end{align*}
is called a \emph{Hamiltonian} system with $n$ degrees of freedom.

\begin{theorem}[Conservation of Energy]
	The total energy $H(\vec x, \vec y)$ of a Hamiltonian system remains constant along its trajectories.
\end{theorem}

\begin{proof}
	The total derivative of the Hamiltonian function $H(\vec x, \vec y)$ along a solution trajectory $\vec x(t), \vec y(t)$ is
	\begin{align*}
		\frac{\mathrm dH}{\mathrm dt} 	&= \frac{\partial H}{\partial \vec x} \cdot \dot{\vec x} + \frac{\partial H}{\partial \vec y} \cdot \dot{\vec y} \\
										&= \frac{\partial H}{\partial \vec x} \cdot \frac{\partial H}{\partial \vec y} - \frac{\partial H}{\partial \vec y} \cdot \frac{\partial H}{\partial \vec x} \\
										&= 0
	\end{align*}
\end{proof}

\subsection{Gradient Systems}
A system of the form
\begin{equation}
	\dot{\vec x} = -\grad{V(\vec x)}
\end{equation}
is called a \emph{gradient system}.

\begin{theorem}
	At regular points of the function $V(\vec x)$, trajectories of a gradient system cross the level surfaces $V(\vec x) = \text{constant}$ orthogonally. And strict local minima of the function $V(\vec x)$ are asymptotically stable equilibrium points of this gradient system.
\end{theorem}

Also, $V(\vec x) - V(\vec x_0)$ is a strict Lyapunov function for this system, where $\vec x_0$ is a strict local minimum of $V(\vec x)$.

\subsection{A relationship between Gradient and Hamiltonian Systems}
They are orthogonal.